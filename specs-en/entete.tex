\section{Overview}
\begin{figure}[!ht]
\input{Images/figure_globale-en.pstex_t}
\caption{Global format overview}
\label{fig:schema_general}
\end{figure}

Figure \ref{fig:schema_general}
represents the general overview of a QPDL document.
We can identify four main parts:
\begin{enumerate}
	\item The markers at the beginning and end of a job: the \textbf{JCL} markers;
	\item The \textbf{PJL header}\footnote{Printer Job Language}
		describes the necessary printing instructions for the job
		(paper thickness, print quality, \ldots);
	\item Begin page and end page records;
	\item The printing bands which represent several lines of dots
		in the form of an array.
\end{enumerate}

The following sections describe each of these parts.


\section{The JCL Markers}
These mark the beginning and end of a print job.
Their values are found in the PPD\footnote{Portable Printer Description}
file associated with the printer under the names \textbf{JCLBegin} and \textbf{JCLEnd}.
\medskip

For example, some printers use these markers:
\texttt{\^{}[\%-12345X} and \texttt{$\backslash$t\^{}[\%12345X}
\footnote{\texttt{$\backslash$t} is the \emph{tab} character (9) and
\texttt{\^{}[} is the \emph{escape} character (27).}.


\section{The PJL Header}
PJL is a language whose specifications are freely available
on the Internet and hence doesn't require any particular comment.

Here is a \emph{non exhaustive} list of the options available:
Papertype, Density, Powersave, Powersavetime, Ret, JamRecovery, Reprint,
Altitude, \ldots.

The last command in the header indicates the start of the QPDL code:
\texttt{@PJL ENTER LANGUAGE = QPDL$\backslash$n}
\footnote{\texttt{$\backslash$n} is the \emph{linefeed (10)} character.}
\medskip

Finally, a small example of a possible JPL header:
\begin{verbatim}
@PJL DEFAULT SERVICEDATE=20060524
@PJL SET PAPERTYPE = BOND
@PJL SET DENSITY = 3
@PJL SET ECONOMODE = ON
@PJL SET POWERSAVE = ON
@PJL SET POWERSAVETIME = 5
@PJL SET RET = OFF
@PJL SET JAMRECOVERY = OFF
@PJL SET REPRINT = ON
@PJL SET ALTITUDE = LOW
@PJL ENTER LANGUAGE = QPDL
\end{verbatim}




\section{Begin page and end page records}


\subsection{The Header}
Preceding all pages is a header that provides indispensable information
for the proper interpretation of the following data.

\begin{table}[!ht]
\centering
\begin{tabular}{| c | c | c |}
\hline
\textbf{Address} & \textbf{Description} & \textbf{Type} \\
\hline
\hline
0x0 & Fixed value $0$ & Byte \\
0x1 & Resolution divided by $100$ (horizontal or vertical ?) & Byte \\
0x2 & Number of copies & 16-Bits BE\footnote{Big endian} \\
0x4 & Paper type (cf. table \ref{tab:liste_papiers}) & Byte \\
0x5 & Paper width if \emph{Custom} if not $0$ & 16-Bits BE \\
0x7 & Paper length if \emph{Custom} if not $0$ & 16-Bits BE \\
0x9 & Paper feeder to use  (cf. table \ref{tab:bac_feuille}) & Byte \\
0xA & Fixed value (?) $0$ & Byte \\
0xB & Duplex ($0$ or $1$) & Byte \\
0xC & Duplex tumble ($0$ or $1$) & Byte \\
0xD & Fixed value (?) $0$ & Byte \\
0xE & Fixed value (?) $1$ & Byte \\
0xF & Fixed value (?) $0$ & 16-Bits BE\\
\hline
\end{tabular}
\caption{New page header description}
\label{tab:entete_page}
\end{table}

\begin{table}[ht]
\centering
\begin{tabular}{| c | c || c | c || c | c || c | c |}
\hline
\textbf{Code} & \textbf {Description} & \textbf{Code} & \textbf {Description} &
\textbf{Code} & \textbf {Description} & \textbf{Code} & \textbf {Description} \\
\hline
\hline
0 & Letter & 1 & Legal & 2 & A4 & 3 & Executive \\
4 & Ledger & 5 & A3 & 6 & Com10 & 7 & Monarch \\
8 & C5 & 9 & DL & 10 & JB4 & 11 & JB5 \\
12 & B5 & 13 & Not listed & 14 & JPost & 15 & JDouble \\
16 & A5 & 17 & A6 & 18 & JB6 & 21 & Custom \\
23 & C6 & 24 & Folio & & & &  \\
\hline
\end{tabular}
\caption{Paper format codes}
\label{tab:liste_papiers}
\end{table}


\begin{table}[ht]
\centering
\begin{tabular}{| c | c || c | c |}
\hline
\textbf{Code} & \textbf{Description} & \textbf{Code} & \textbf{Description} \\
\hline
\hline
1 & Auto & 2 & Manual \\
3 & Multi & 4 & Top \\
5 & Bottom & 6 & Envelopes \\
7 & Third  & & \\
\hline
\end{tabular}
\caption{Paper feeder codes}
\label{tab:bac_feuille}
\end{table}

The description of this header is given in table \ref{tab:entete_page}.

The significance of some fields is not clear.
For example, in the case where the paper type is defined as
\emph{custom},
the two 16-Bit words following this field define the paper dimensions;
but the algorithm to pass a centimeter value in this field is
not yet known.

The values \emph{0xA, 0xD - 0x10} surely
have a defined purpose, but this it not yet known.

One hypothesis is
that some of these bytes are used to define the horizontal resolution (with the
resolution field actually being the vertical resolution), for those printers
that have nonsymmetric resolutions.
\bigskip

It appears that the real resolution calculation algorithm is:

\begin{equation*}
\frac{\textrm{bits32--63}(\textrm{resolution} \times \textrm{0x}51EB851F)
\times 32 + \frac{\textrm{resolution}}{\textrm{0x}80000000}}{1024}
\end{equation*}

Therefore, with the usual resolutions of 300dpi, 600dpi, or 1200dpi
we get 3, 6, or 12.  This is equivalent to dividing the resolution by $100$.


\subsection{The page footer}
The description of the page footer is given in table \ref{tab:fin_page}.

\begin{table}[!ht]
\centering
\begin{tabular}{| c | c | c |}
\hline
\textbf{Address} & \textbf{Description} & \textbf{Type} \\
\hline
\hline
0x0 & Fixed value $1$ & Byte\\
0x1 & Number of copies & 16-Bits BE \\
\hline
\end{tabular}
\caption{Description of the page footer}
\label{tab:fin_page}
\end{table}
