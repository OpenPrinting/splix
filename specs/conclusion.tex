\section{Conclusion}
Le langage SPL2 est donc un langage de description de document point par point.
La très grande quantité de données engendrée par cette approche est modérée
par l'utilisation d'algorithmes de compressions dont le plus couramment utilisé
s'occupe de recoder des séquences de données par rapport aux données
précédentes déjà décompressées.

Associé à la réorganisation des bandes en colonnes, cela permet d'obtenir des
taux de compressions allant jusqu'à $99\%$ avec une moyenne de $50\%$ pour un
document standard.
