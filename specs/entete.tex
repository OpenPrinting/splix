\section{Une première approche}
\begin{figure}[!ht]
\input{Images/figure_globale.pstex_t}
\caption{Schéma global du format}
\label{fig:schema_general}
\end{figure}

La figure \ref{fig:schema_general} représente le schéma général d'un
document QPDL. On peut y distinguer quatres parties différentes :
\begin{enumerate}
	\item Les balises de début et de fin d'une tâche : les balises 
		\textbf{JCL}~;
	\item L'en-tête \textbf{PJL}\footnote{Printer Job Language} décrivant
		les réglages de l'imprimante nécessaires pour l'impression
		de la tâche (taille du papier, qualité de l'impression,
		\ldots)~;
	\item Les blocs d'information de début et de fin de page~;
	\item Les bandes d'impression qui représentent plusieurs lignes de 
		points de la page sous forme d'une carte de points.
\end{enumerate}

Les prochaines sections décriront tour à tour chacunes de ces parties.




\section{Les balises JCL}
Ces balises marquent le début et la fin d'une impression. Leurs valeurs se
trouvent dans le fichier PPD\footnote{Portable Printer Description} associé
à l'imprimante sous le nom \textbf{JCLBegin} et \textbf{JCLEnd}.
\medskip

Par exemple, certaines imprimantes utilisent les balises suivantes : 
\texttt{\^{}[\%-12345X} et \texttt{$\backslash$t\^{}[\%12345X}
\footnote{\texttt{$\backslash$t} est le caractère \emph{tabulation} (9) tandis 
que \texttt{\^{}[} est le caractère d'\emph{échappement} (27).}.




\section{L'en-tête PJL}
PJL est un langage dont les specifications sont disponibles librement sur 
Internet et sa mise en \oe uvre ne nécessite aucune remarque particulière.

Voici ici une liste \emph{non exhaustive} des options utilisables :
Papertype, Density, Powersave, Powersavetime, Ret, JamRecovery, Reprint,
Altitude, \ldots.

La dernière commande sera celle qui donne la main au code QPDL : 
\texttt{@PJL ENTER LANGUAGE = QPDL$\backslash$n}
\footnote{\texttt{$\backslash$n} est le caractère \emph{retour chariot (10).}}
\medskip

Enfin, un petit exemple d'un en-tête complet JPL possible :
\begin{verbatim}
@PJL DEFAULT SERVICEDATE=20060524
@PJL SET PAPERTYPE = BOND
@PJL SET DENSITY = 3
@PJL SET ECONOMODE = ON
@PJL SET POWERSAVE = ON
@PJL SET POWERSAVETIME = 5
@PJL SET RET = OFF
@PJL SET JAMRECOVERY = OFF
@PJL SET REPRINT = ON
@PJL SET ALTITUDE = LOW
@PJL ENTER LANGUAGE = QPDL
\end{verbatim}




\section{Les blocs d'information de début et de fin de page}


\subsection{L'en-tête}
Avant tout début de page se situe un en-tête fournissant des informations
indispensables à la bonne interprétation des données.

\begin{table}[!ht]
\centering
\begin{tabular}{| c | c | c |}
\hline
\textbf{Adresse} & \textbf{Description} & \textbf{Type} \\
\hline
\hline
0x0 & Valeur fixe $0$ & Octet \\
0x1 & Résolution divisée par $100$ (horizontale ou verticale ?) & Octet \\
0x2 & Nombre de copies & 16-Bits BE\footnote{Big endian} \\
0x4 & Type de papier (cf. table \ref{tab:liste_papiers}) & Octet \\
0x5 & Largeur du papier si \emph{Custom} sinon $0$ & 16-Bits BE \\
0x7 & Longueur du papier si \emph{Custom} sinon $0$ & 16-Bits BE \\
0x9 & Bac de feuille à utiliser (cf. table \ref{tab:bac_feuille}) & Octet \\
0xA & Valeur fixe (?) $0$ & Octet \\
0xB & Duplex ($0$ ou $1$) & Octet \\
0xC & Duplex tumble ($0$ ou $1$) & Octet \\
0xD & Valeur fixe (?) $0$ & Octet \\
0xE & Valeur fixe (?) $1$ & Octet \\
0xF & Valeur fixe (?) $0$ & 16-Bits BE\\
\hline
\end{tabular}
\caption{Description de l'en-tête d'une nouvelle page}
\label{tab:entete_page}
\end{table}

\begin{table}[ht]
\centering
\begin{tabular}{| c | c || c | c || c | c || c | c |}
\hline
\textbf{Code} & \textbf {Description} & \textbf{Code} & \textbf {Description} &
\textbf{Code} & \textbf {Description} & \textbf{Code} & \textbf {Description} \\
\hline
\hline
0 & Letter & 1 & Legal & 2 & A4 & 3 & Executive \\
4 & Ledger & 5 & A3 & 6 & Com10 & 7 & Monarch \\
8 & C5 & 9 & DL & 10 & JB4 & 11 & JB5 \\
12 & B5 & 13 & Not listed & 14 & JPost & 15 & JDouble \\
16 & A5 & 17 & A6 & 18 & JB6 & 21 & Custom \\
23 & C6 & 24 & Folio & & & &  \\
\hline
\end{tabular}
\caption{Codes associés aux différents formats de papier}
\label{tab:liste_papiers}
\end{table}


\begin{table}[ht]
\centering
\begin{tabular}{| c | c || c | c |}
\hline
\textbf{Code} & \textbf{Description} & \textbf{Code} & \textbf{Description} \\
\hline
\hline
1 & Auto & 2 & Manuel \\
3 & Multi & 4 & Haut \\
5 & Bas & 6 & Enveloppes \\
7 & Troisième & & \\
\hline
\end{tabular}
\caption{Codes associés aux différents types de sources de feuilles}
\label{tab:bac_feuille}
\end{table}

La description de cet en-tête est donnée dans la table \ref{tab:entete_page}.

Certains champs ont encore des significations peu claires. Par exemple, dans
le cas où le type de papier est défini comme étant \emph{personnalisé}, 
les deux mots de 16-Bits qui suivent ce champ définissent les dimensions du
papier mais l'algorithme permettant de passer d'une dimension en centimètre
à ce champ n'est pas encore connu. Les valeurs \emph{0xA, 0xD - 0x10} ont 
sûrement une utilité propre mais n'est pas encore connue. Une hypothèse serait
que certains de ces octets soient utilisés pour définir une résolution
horizontale (le champ de résolution serait alors la résolution verticale), 
pour le cas des imprimantes qui peuvent avoir des résolutions non symétriques.
\bigskip

Il semblerait que le véritable algorithme de calcul de la résolution soit :

\begin{equation*}
\frac{\textrm{bitsDe32À63}(\textrm{résolution} \times \textrm{0x}51EB851F)
\times 32 + \frac{\textrm{résolution}}{\textrm{0x}80000000}}{1024}
\end{equation*}

Cependant, avec les résolutions usuelles de 300ppp, 600ppp ou 1200ppp, on
obtient 3, 6 ou 12 ; ce qui est équivalent à diviser la résolution par $100$.


\subsection{Le pied de page}
La description de la fin d'une page est donnée dans le tableau
\ref{tab:fin_page}.

\begin{table}[!ht]
\centering
\begin{tabular}{| c | c | c |}
\hline
\textbf{Adresse} & \textbf{Description} & \textbf{Type} \\
\hline
\hline
0x0 & Valeur fixe $1$ & Octet \\
0x1 & Nombre de copies & 16-Bits BE \\
\hline
\end{tabular}
\caption{Description de la fin d'une page}
\label{tab:fin_page}
\end{table}
