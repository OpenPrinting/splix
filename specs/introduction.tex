Certaines imprimantes utilisent parfois un format de données baptisé 
\emph{SPL2} ou \emph{QPDL}. Malheureusement, celui-ci étant propriétaire et 
fermé, il était impossible de développer des pilotes capable d'exploiter
leur langage natif pour obtenir, généralement, des performances meilleures
(il n'est alors plus nécessaire, pour l'imprimante, de passer par un 
émulateur). 

C'est pourquoi, j'ai étudié de près ce langage. Après plusieurs jours 
d'analyses, voici le résultat de mes analyses.
\medskip

Notez que cette documentation n'est pas \emph{officielle} et donc que 
son contenu peut être incomplet ou erroné. \textbf{L'auteur ne peut être 
tenu pour responsable des conséquences de l'utilisation du contenu de ce 
document en cas de dysfonctionnement ou de détérioration du matériel.}
\medskip

Enfin, si vous êtes en possession d'informations susceptibles de le corriger
ou de le compléter, n'hésitez pas à en faire part à l'auteur (Plus
d'informations sur le site du projet : \url{http://splix.sf.net} ou
\url{programmationATap2c.com}).
