Some printers use a data format called \emph{SPL2} or \emph{QPDL}.
Unfortunately, since this is a proprietory closed format, it was
impossible to develop drivers able to exploit the printer's native
language to obtain, generally, better performance
(by avoiding the use of an emulator in order to print).
%???(il n'est alors plus nécessaire, pour l'imprimante, de passer par un 
%???emulateur). 

This is why, I studied this language closely. After several days
of analysis, here are the results of my work.
\medskip

Note that this documentation is \emph{unofficial} and that its
contents may be incomplete or erroneous. \textbf{The author cannot
be held responsible for any consequences of the use of the contents
of this document in the event of failure or damage to your equipment.}
\medskip

Lastly, if you are in possession of more correct or supplemental
information,
do not hesitate to share it with
the author (more information at the project site:
\url{http://splix.ap2c.org} or \url{programmationATap2c.com}).
